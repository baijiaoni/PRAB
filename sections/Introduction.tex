FAIR is a new international accelerator facility under construction at GSI. It is aiming at providing high-energy beams of ions of all elements from hydrogen to uranium with high intensities, as well as beams of rare isotopes and of antiprotons~\cite{eschke_international_2005, noauthor_fair_2011}. The FAIR facility in its start version will consist of three circular accelerators (short: ring), the SIS100, the CR and the HESR. In addition, the GSI accelerator facility complements the planned accelerators of FAIR, which comprises the SIS18, the ESR and the CRYRING. Bunches are required to be transferred into radio frequency (rf) buckets among GSI and FAIR rings by means of the bunch-to-bucket (B2B) transfer method. \textcolor{red}{An implementation of the B2B transfer from the SIS18 to the ESR exists, the phase difference between the two rf systems is measured based on the direct transmission of the analog rf signals of the two rings to a central station. However the direct transmission is undesired for the new FAIR accelerator complex, which will be around six times larger than the existing GSI site. In order to avoid the direct transmission, a center clock can be used, which provides a reference clock to the rf systems of the two rings. The phase difference between the rf system and the reference clock is measured at each ring locally and the measurement results are transferred to a central station to calculate the phase difference. Coincidentally, the FAIR Bunchphase Timing System (BuTiS)~\cite{moritz_butisdevelopment_2006} can provide such a reference clock for the B2B transfer, which serves as a campus-wide clocks distribution system for the low level radio frequency (LLRF) system ~\cite{klingbeil_new_2011} with sub nanosecond resolution and stability over distances of several hundred meters~\cite{moritz_f-cs-rf-14e_2012}. In addition, the FAIR new timing system, the General Machine Timing (GMT) system~\cite{beck_general_2013}, is based on the sub-nanosecond synchronization White Rabbit (WR) network~\cite{beck_white_2011}, which can transfer the measurement results deterministically. Therefore, a new FAIR B2B transfer system based on the GMT system, the LLRF system and the BuTiS is required. }

\textcolor{red}{The concept of the B2B transfer was first introduced in the early 1980s at the European Organization for Nuclear Research (CERN) for the beam transfer from the Proton Synchrotron Booster (PSB) to the PS~\cite{garoby_cern-ps-rf-note-84-6_1984} and the B2B transfer has been used for almost three decades around the world. The FAIR B2B transfer is comparable with the CERN B2B transfer. At CERN, primary beams of ions of all elements can be transferred among rings as at FAIR, e.g. the Large Hadron Collider (LHC) is supplied with \SI{7}{TeV} high energy proton beams from the injector chain Linac2 - PSB - PS -  Super Proton Synchrotron (SPS) and with \SI{2.76}{TeV/u} high energy heavy ion beams from the injection chain Linac3 - Low Energy Ion Ring (LEIR) - PS - SPS  ~\cite{noauthor_cern_nodate}. In addition, the CERN B2B transfer system transfers the secondary beams to rf buckets, e.g. a proton beam that comes from the PS is fired into an antiproton (pbar) target to produce the antiproton beam and the antiproton beam is transferred into Antiproton Decelerator (AD). However, the CERN B2B transfer is based on the direct transmission of the analog signal (e.g. the revolution signal of a reference bucket) from the target ring to the source ring as the GSI existing B2B transfer implementation, which is used for the bucket counting and synchronization. The unstable (e.g. the thermal drift) of every analog signal transmission delay along coaxial cables or optical fibers is compensated individually. In order to synchronize two rings, the beam of the source ring is moved to an off-momentum position by adjusting the cavities frequency (the magnetic field is constant). A periodic beating of the phase between the revolution signals of the two rings is observed. The relative azimuthal position between the two rings is measured and the beam is moved back to the reference momentum when it reaches the correct azimuth ~\cite{damerau_lecture_2017}. The synchronization process between the PS and the SPS takes about \SI{50}{\ms}~\cite{ferrand_cern-acc-note-2015-0025_2015} and that between the SPS and the LHC takes about \SI{30}{\ms}~\cite{baudrenghien_sps_1998}}. 

******** add something
%The Japan Proton Accelerator Complex (J-PARC) uses the B2B transfer to transfer proton beams from the Rapid Cycle Synchrotron (RCS) to the Main Ring (MR) to gain higher energy for the further production of the desired secondary particle beam~\cite{noauthor_j-parc_2016}. In the injection period of the MR, the rf system provides rf signals of a fixed frequency. The information (the phase of the MR rf system, the empty bucket tag) sent from the MR to the RCS. The positions of bunches in the RCS are controlled during acceleration by using the phase information from the MR so that two bunches are in the proper phase at the top energy. The tag information gives the proper timing for triggering the kickers at the time of the empty buckets in the MR ~\cite{tamura_synchronization_2006}. 



In the following the requirements for the FAIR B2B transfer system is listed. In Sec. ~\ref{concept} the concept of the FAIR B2B transfer system is introduced together with the basic procedure. The FAIR B2B transfer system focus first of all on the transfer from the SIS18 to the SIS100. Hence, the Sec. ~\ref{dynamics} is concerned with the analysis of the two synchronization methods from the beam dynamics for the B2B transfer from the SIS18 to the SIS100. In Sec.~\ref{timing}, the timing constraints of the system is presented. The different trigger scenarios of the SIS18 extraction and SIS100 injection kickers are discussed in Sec. ~\ref{kicker}. Afterwards the application of the system for FAIR use cases with the frequency beating method is presented in Sec. ~\ref{application}.


