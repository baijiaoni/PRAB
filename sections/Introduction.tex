The bunch-to-bucket (B2B) transfer is a process that bunches circulating in the source circular accelerator (short: ring) are transferred into the center of rf buckets of the target ring. This transfer method has already been used in several accelerator institutes worldwide for specific purposes. CERN, the European Organization for Nuclear Research, makes use of the B2B transfer to produce high energy beams for scientific research. The Large Hadron Collider (LHC) is supplied with \SI{7}{TeV} high energy proton beam from the injector chain Linac2 - Proton Synchrotron (PS) Booster - PS -  Super Proton Synchrotron (SPS) and with \SI{2.76}{TeV/u} high energy heavy ion beam from the injection chain Linac3 - Low Energy Ion Ring (LEIR) - PS - SPS  ~\cite{noauthor_cern_nodate}. The transfer among CERN rings is realized by the B2B transfer. The CERN B2B transfer system is based on the direct transmission of the revolution frequency signal from the target to the source ring around the campus for the bucket counting and synchronization. In order to synchronize two rings, beam is moved to off-momentum by adjusting frequency (magnetic field is constant). The phase difference between two rings varies periodically. Then the azimuth error between two rings is measured and beam is moved back to the reference momentum when beam is at correct azimuth. The complete synchronization process takes about \SI{500}{\ms} ~\cite{damerau_lecture_2017}. The Japan Proton Accelerator Complex (J-PARC) uses the B2B transfer to transfer proton beams from the Rapid Cycle Synchrotron (RCS) to the Main Ring (MR) to gain higher energy for the further production of the desired secondary particle beam~\cite{noauthor_j-parc_2016}. In the injection period of the MR, the rf system provides rf signals of a fixed frequency. The information (the phase of the MR rf system, the empty bucket tag) sent from the MR to the RCS. The positions of bunches in the RCS are controlled during acceleration by using the phase information from the MR so that two bunches are in the proper phase at the top energy. The tag information gives the proper timing for triggering the kickers at the time of the empty buckets in the MR ~\cite{tamura_synchronization_2006}. 

FAIR is a new international accelerator facility under construction at GSI. It is aiming at providing high-energy beams of ions of all elements from hydrogen to uranium with high intensities, as well as beams of rare isotopes and of antiprotons~\cite{eschke_international_2005, noauthor_fair_2011}.  The FAIR accelerators will be supplied with ion beams by the GSI accelerator facility, which comprises the injectors for the FAIR accelerators. The injection chain consists of the linear accelerator UNILAC and the heavy ion synchrotron SIS18. In addition, the GSI accelerator facility comprises the ESR and the CRYRING@ESR (short: CRYRING), which complement the planned accelerators of FAIR. The FAIR accelerator complex in its start version will consist of the SIS100, the CR and the HESR. For FAIR, the intermediate charge state ions are used to increase the beam intensity by reducing space charge at SIS18, causing the larger mass-to-charge ratio. However, the intermediate charge state ion beams can not be accelerated to high enough energy at the existing SIS18 due to the constraints of the magnetic rigidity. On the other hand, the intermediate charge state ion beams cause the dynamic vacuum challenge by their significantly enhanced cross section for ionization and their high potential for generating ion desorption driven vacuum instabilities. Hence, the SIS100 with a larger magnetic rigidity is required for the further acceleration. Beside, the perfect control over the dynamic vacuum at SIS100 realizes smaller cross sections for charge exchange and the special lattice design optimizes best control of the ionization beam loss ~\cite{nolden_collector_2006, bar_technical_2013}. Furthermore, with the double ring facility, high average intensity heavy ion beams can be provided with the help of the beam stacking by the multiple injection. The CR will accumulate the secondary beams and improve their quality by stochastic cooling. The storage ring HESR will host a large fraction of the experiment platforms with a variety of different experiments~\cite{spiller_fair_2006, steck_advanced_2009}. The high energy antiproton/rare isotope beam chain is composed of the UNILAC - SIS18 - SIS100 -antiproton target/superconducting fragment separator- CR - HESR and the lower energy chain cascades of UNILAC - SIS18 -fragment separator- ESR - CRYRING.

Although an implementation of the B2B transfer from the SIS18 to the ESR exists, this solution is not applicable for the new FAIR accelerator complex, because it is realized based on the GSI control system ~\cite{krause_gsi_1991}, which will be replaced by a new control system for FAIR. The FAIR control system is based on the sub-nanosecond synchronization White Rabbit (WR) network. Besides, the existing B2B transfer does not support the B2B transfer between two rings with an integer circumference ratio. 

In the following the requirements for the FAIR B2B transfer system is listed. In Sec. ~\ref{concept} the concept of the FAIR B2B transfer system is introduced together with the basic procedure. The FAIR B2B transfer system focus first of all on the transfer from the SIS18 to the SIS100. Hence, the Sec. ~\ref{dynamics} is concerned with the analysis of two synchronization methods from the beam dynamics for the B2B transfer from the SIS18 to the SIS100. In Sec.~\ref{timing}, the timing constraints of the system is presented. The different trigger scenarios of the SIS18 extraction and SIS100 injection kickers are discussed in Sec. ~\ref{kicker}. Afterwards the application of the system for FAIR use cases with the frequency beating method is presented in Sec. ~\ref{application}.


