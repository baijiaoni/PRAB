Development of the timing system for the Bunch-to-Bucket transfer between the FAIR accelerators plays a significant important role for the realization of the FAIR B2B transfer system and the further practical application of the system to all FAIR use cases. The new FAIR B2B transfer system is developed based on the FAIR technical basis, the FAIR timing and control system and the low-level radio frequency system. The system achieves the most FAIR B2B transfers with a tolerable bunch-to-bucket injection center mismatch (e.g. $\pm 1^\circ$) and within an upper bound time (e.g. \SI{10}{\ms}). It supports both the phase shift and frequency beating methods. It is flexible to support the beam transfer between two rings with different ratios in their circumference and several B2B transfers running at the same time, e.g. the B2B transfer from the SIS18 to the SIS100 and at the same time the B2B transfer from the ESR to the CRYRING. It is capable to transfer beam of different ion species from one machine cycle to another. It supports the beam transfer via the FRS, the pbar target and the Super-FRS. It allows various complex bucket filling patterns. In addition, it coordinates with the MPS system, which protects the SIS100 and subsequent accelerators or experiments from beam induced damage. 

For all primary beam transfers of FAIR use cases, the B2B transfer with the bunch-to-bucket injection center mismatch less than $\pm1^\circ$ and within the required B2B transfer time \SI{10}{\ms} can be achieved. However, it works for the FAIR use cases of the secondary beams generated by the pbar target, the FRS or the Super-FRS only together with specific beam accumulation methods.

The realistic test of the system on FAIR accelerators will be done at the end of 2018, because many FAIR technical basis and rings are still under construction.

