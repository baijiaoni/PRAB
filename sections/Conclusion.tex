The concept of the FAIR B2B transfer system and its application for all FAIR use cases were presented, as well as the systematic investigation from the timing and the beam dynamics perspectives. The new FAIR B2B transfer system fulfills the requirements for FAIR. 

For the B2B transfer from the SIS18 to the SIS100 with the phase shift method, the sinusoidal rf frequency modulation is a better choice compared with the same periodical parabolic modulation. It needs \SI{7}{ms} for the SIS18 \SI{200}{MeV/u} $U^\mathit{28+}$ and \SI{50}{ms} for the SIS18 \SI{4}{GeV/u} $H^+$ to achieve the phase shift of $\pi$ in order to guarantee a bucket area factor larger than $80\%$ and an adiabaticity smaller than $10^{-4}$.

Two test setups were built to verify the timing constraints. The first test setup was used to characterize the WR network for the B2B transfer. If for instance one lost frame is tolerable every two month, the maximum 38 WR switch layers can be used between the B2B related SCUs and the DM and the maximum 8 WR switch layers can be used between the B2B related SCUs. In the second test setup, the firmware of the FAIR B2B transfer system was evaluated, running on the soft CPU, LatticeMico32, of the SCUs. The measurement results show that the firmware running on the LatticeMico32 of the SCUs meets the requirement of the timing constraints.

For all primary beam transfers of FAIR use cases, the B2B injection center mismatch within $\pm1.2^\circ$ is acceptable. However, for the FAIR use cases of the secondary beams, the mismatch is as large as $\pm41.5^\circ$. 

The most important investigations for the FAIR B2B transfer system were discussed in this manuscript. However, there are still some investigations which are required for the final system operation. The magnetic horn after the pbar target has to be synchronized with the antiproton beam to the ``\SI{}{us}`` order of magnitude. The bunch compressor of the SIS100 has to be synchronized the beam extraction. Finally, for the FAIR use cases of the secondary beam with the B2B injection center mismatch larger than $\pm41.5^\circ$, the FAIR B2B transfer system with specific beam accumulation methods (e.g. the barrier bucket or the unstable fixed point accumulation) has to be checked. 

The realistic test of the system on FAIR accelerators will be done at the end of 2018, because many FAIR technical basis and rings are still under construction.

