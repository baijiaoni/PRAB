The B2B transfer needs a fast beam extraction and injection, which extracts and injects the beam in a single-turn. Hence, a pulsed kicker magnet must be used with rapid rise time and fall time and the variable pulse flat-top ~\cite{petzenhauser_concept_2016}. Fig. ~\ref{SIS18_kicker} shows the schematic diagram of a kicker magnet. The energy storage module is charged with a high voltage power supply. It will be discharged via the transmission cable and the kicker magnet by using the pulse start switch. The length of the flat-top can be modified by switching on the stop switch in correlation with the pulse start switch. Before the increase of the magnetic field, there exist a preparation time for the kicker magnet. The kicker control electronic produces the ignition signal to switch on/off two switches. Generally a preparation time of FAIR kickers is within the \SI{5}{}$-$\SI{10}{\micro\second} range. Compared with the FAIR rf frequency in the \SI{}{MHz} range, a preparation time is not negligible.
\begin{figure}[!htb]
   \centering   
   \includegraphics*[width=85mm]{SIS18_kicker.jpg}
   \caption{Schematic diagram of a kicker magnet.}
   \label{SIS18_kicker}
\end{figure}


The SIS18 extraction kicker consists of nine kicker magnets. In the existing topology, five kicker magnets are evenly distributed in a tank (the $1^{st}$ tank) and the other four kicker magnets are evenly distributed in another tank (the $2^{nd}$ tank). The width of each kicker magnet is \SI{0.25}{m} and the distance between two kicker magnets is \SI{0.09}{m}. The distance between the two tanks is \SI{19.17}{m} ~\cite{ros_sis18_2008}. The rise time of the kicker magnet $t_\mathit{rise}$ is approximately \SI{90}{ns} ~\cite{blell_f-ds-ie-03e_2014}. Bunches are firstly kicked by kicker magnets in the $1^{st}$ tank and than kicked by the
kicker magnets in the $2^{nd}$ tanks to the transfer line. The SIS100 injection kicker consists of six kicker magnets, which are evenly distributed in a common tank. The width of each kicker magnet is \SI{0.22}{m} and the distance between two magnets is \SI{0.23}{m}. The rise time of the kicker magnet $t_\mathit{rise}$ is \SI{130}{ns} ~\cite{blell_f-ds-ie-03e_2014}. 

Three ion beams, \SI{4}{GeV/u} $H^+$, \SI{200}{MeV/u} $U^{28+}$ and \SI{970}{MeV/u} $U^{73+}$, are used to check the instantaneous trigger of kicker magnets in a common tank, because these ion species have the most stringent requirements. In order to instantaneous trigger of kicker magnets in one tank, the bunch tail passing time from one side to another side of a tank plus the kicker rise time must be shorter than the bunch gap, namely

\begin{equation}
		\frac{tank\_length}{\beta c}+rise\_time<bunch\_gap
\end{equation}

For the SIS18 and SIS100 kickers, there are the following technical constraints.
\begin{itemize}
	\item The SIS18 kicker magnets in each tank can be triggered simultaneously when the bunch gap is at least $25\%$ of the cavity rf period. 

	\item The six SIS100 injection kicker magnets can be fired instantaneously for all ion beams, when the bunch gap is at least $35\%$ of the cavity rf period.
\end{itemize}

In addition, the maximum triggering delay between SIS18 kicker magnets in the two tanks occurs when the kicker magnets in the $1^{st}$ tank are triggered simultaneously when the tail of a bunch passes by the $1^{st}$ tank completely and the kicker
magnets in the $2^{nd}$ tank are simultaneously triggered \SI{90}{ns} before the head of the next bunch to be extracted passes by it. The minimum triggering delay occurs when the kicker magnets in the $1^{st}$ tank are simultaneously triggered \SI{90}{ns} before the head of the bunch to be extracted passes by it and the kicker magnets in the $2^{nd}$ tank are simultaneously triggered when the last bunch passes by the $2^{nd}$ tank. According to the calculation, the kicker magnets in the $2^{nd}$ tank can be simultaneously triggered a fixed delay to the simultaneous trigger of the kicker magnets in the $1^{st}$ tank for all ion beams, when the bunch gap is at least $25\%$ of the cavity rf period, e.g. \SI{80}{\ns}.