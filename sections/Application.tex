%%%%%%%%%%%%%%%%%%%%%%% publication

\begin{table*}[!hbp]
\newcommand{\tabincell}[2]{\begin{tabular}{@{}#1@{}}#2\end{tabular}}
\caption{Application of the FAIR B2B transfer system for FAIR accelerators}
\label{use_application}
\begin{center}

    \begin{tabular}{| c | p{4cm}| p{2cm} | c | c |p{7cm} |}
    \hline
%\rowcolor[gray]{0.5}
     	No. & \multicolumn{1}{|c|}{FAIR use cases} & B2B injection center mismatch& $C^l/C^s$ & $f^s_\mathit{rev}/f^l_\mathit{rev}$  & \multicolumn{1}{|c|}{Remark} \\ \hline
1&	Four batches, each of two SIS18 $U^{28+}$ bunches (\SI{200}{MeV/u}) $\rightarrow$ eight out of ten SIS100 buckets	&	$\pm0.4^\circ$ 	& 5 & & \multirow{5}{*}{\parbox{7cm}{The FAIR use cases 1-5 have the B2B injection center mismatch smaller than the upper bound $\pm1^\circ$, because the circumference ratio between two rings is an integer or close to an integer.}}\\ \cline{1-5}
2&Four batches, each of one SIS18 $H^{+}$ bunch (\SI{4}{GeV/u}) $\rightarrow$ four out of ten SIS100 buckets	&	$\pm0.4^\circ$		& 5 & &	\\ \cline{1-5}
3&Two of four SIS18 bunches (\SI{30}{MeV/u}) $\rightarrow$ two ESR buckets			&	$\pm0.5^\circ$	& 2-0.003& &	\\ \cline{1-5}
4&One SIS18 $H^{+}$ bunch (\SI{400}{MeV/u}) $\rightarrow$ one ESR bucket	&	$\pm0.5^\circ$	& 2-0.003& &  \\ \cline{1-5}
5&One ESR bunch (\SI{30}{MeV/u}) $\rightarrow$ one CRYRING bucket				&	$\pm0.5^\circ$	& 2-0.003& &	\\ \hline
6&One CR antiproton bunch (\SI{3}{GeV/u}) $\rightarrow$ one HESR bucket	&$\pm1.2^\circ$ &2.6-0.003 & & \multirow{2}{*}{\parbox{7cm}{The B2B injection center mismatch is just beyond the specification, but it is still acceptable. Although the circumference ratio between two rings is far away from an integer.}}\\ \cline{1-5}
7&One CR RIB bunch (\SI{740}{MeV/u}) $\rightarrow$ one HESR bucket	&$\pm1.2^\circ$ &2.6-0.003 & &\\ \hline
8&One SIS100 $H^{+}$ bunch (\SI{28.8}{GeV/u}) $\xrightarrow[\text{target}]{\text{par}}$ (\SI{3}{GeV/u}) one CR bucket  &$\pm41.5^\circ$& not applicable & arbitrary& \multirow{5}{*}{\parbox{7cm}{The B2B injection center mismatch is far beyond the specification, because the energy ratio before and after targets is arbitrary. (The FAIR use case No. 8 is close to the specification and still acceptable by coincidence.)}}\\ \cline{1-5}
9&One SIS100 RIB bunch (\SI{1.5}{GeV/u}) $\xrightarrow[\text{}]{\text{Super-FRS}}$ (\SI{740}{MeV/u}) one CR bucket 	&	$\pm2.1^\circ$	& not applicable & arbitrary&\\ \cline{1-5}

10&One SIS18 RIB bunch (\SI{550}{MeV/u}) $\xrightarrow[\text{}]{\text{FRS}}$ (\SI{400}{MeV/u}) one ESR bucket &$\pm31.2^\circ$	& not applicable & arbitrary&		\\ \hline

    \end{tabular}
\end{center}
\end{table*}

For the FAIR B2B transfer system, both the phase shift and frequency beating methods are applicable. However, many FAIR accelerator pairs can only use the frequency beating method because of the non-integer ratio of the circumference between two ring accelerators. Besides, FAIR has many use cases of the B2B transfer that the extraction and injection beam have different energy because of targets installed between two ring accelerators. In this case, the beam revolution frequency ratio between the small and large accelerators is used to calculate the synchronization frequencies instead of the circumference ratio between the large and small accelerators. 

Tab. ~\ref{use_application} shows that for all primary beam transfers of FAIR use cases, the B2B transfer with the bunch-to-bucket injection center mismatch less than $\pm1^\circ$ and within the required B2B transfer time \SI{10}{\ms} can be achieved, because the circumference ratio between two rings is an integer or close to an integer. However, the system is also required for the FAIR use cases that the secondary beams are generated by the pbar target, the FRS or the Super-FRS with an arbitrary energy ratio between the primary and secondary beams. For the rare isotope beam (RIB) transfer from the SIS100 to the CR via the Super-FRS with the \SI{1.5}{GeV/u} primary beam energy and the \SI{740}{MeV/u} secondary beam energy, the bunch-to-bucket injection center mismatch is only $\pm2.1^\circ$ by coincidence. For the antiproton B2B transfer from the SIS100 to the CR via the pbar target and the RIB transfer form the SIS18 to the ESR via the FRS, the bunch-to-bucket injection center mismatch is as large as $\pm40^\circ$, which is far beyond the upper bound injection center mismatch. For these FAIR use cases, the FAIR B2B transfer system can work together with specific beam accumulation methods, e.g. the barrier bucket or the unstable fixed point accumulation.



