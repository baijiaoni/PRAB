The FAIR B2B transfer system shall fulfill the requirements for all FAIR use cases. It should achieve the FAIR B2B transfers with a tolerable bunch-to-bucket injection center mismatch (e.g. $\pm 1^\circ$ for most FAIR use cases) and within an upper bound time (e.g. \SI{10}{\ms} for most FAIR use cases). It must support both the phase shift and frequency beating methods (see Sec. ~\ref{concept}). It must be flexible to support the beam transfer between two rings with an arbitrary ratio in their circumference and several B2B transfers running at the same time, e.g. the B2B transfer from the SIS18 to the SIS100 and at the same time the B2B transfer from the ESR to the CRYRING. It should be capable to transfer beam of different ion species from one machine cycle to another. It supports the beam transfer via the the antiproton (pbar) target, the fragment separator (FRS) and the superconducting fragment separator (Super-FRS). It must support various complex bucket filling patterns, e.g. eight out of ten SIS100 buckets are filled by four SIS18 batches, each of two bunches. In addition, it  should coordinate with the machine protection system, which protects the SIS100 and subsequent accelerators or experiments from beam induced damage. 