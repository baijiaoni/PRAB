\textcolor{red}{The FAIR B2B transfer system should support primary beam transfers with a tolerable bunch-to-bucket injection center mismatch ($\pm 1^\circ$ for most FAIR use cases) and within an upper bound time (\SI{10}{\ms} for most FAIR use cases), e.g. bunches are transferred from the SIS18 to the SIS100, from the SIS18 to the ESR  and further to the CRYRING, from the CR to the HESR. It must be flexible to support the beam transfer between two rings with an arbitrary ratio in their circumferences, e.g. the circumference ratio between the SIS100 and the SIS18 is $5$, an integer, between the SIS18 and the ESR is $2 - 0.003$, close to an integer and between the HESR and the CR is $2.6 - 0.003$, far away from an integer. In addition, both the phase shift and frequency beating methods (see Sec. ~\ref{concept}) must be available. }

\textcolor{red}{The difficulty is to transfer beams separated by fragmentation into the target ring with a tolerable bunch-to-bucket injection center mismatch (e.g. $\pm ??^\circ$ mismatch for the ESR barrier bucket injection), because secondary beams have arbitrary energy compared with primary beams. For example, one \SI{28.8}{GeV/u} proton bunch extracted from the SIS100 is transferred to the pbar target, producing a \SI{3}{GeV/u} antiproton bunch. Further the antiproton bunch is injected into a CR bucket. One \SI{550}{MeV/u} heavy ion bunch extracted from the SIS18 is transferred to a target and a \SI{400}{MeV/u} rare isotope beam (RIB) bunch is separated by the fragment separator (FRS) and further transferred into an ESR bucket. One \SI{1.5}{GeV/u} heavy ion bunch extracted from the SIS100 is transferred to a target and a \SI{740}{MeV/u} RIB bunch is separated by a superconducting FRS (Super-FRS) and further transferred into a CR bucket.}


Several B2B transfers running at the same time are required, e.g. the B2B transfer from the SIS18 to the SIS100 and at the same time the B2B transfer from the ESR to the CRYRING. It should be capable to transfer beam of different ion species from one machine cycle to another. Various complex bucket filling patterns should be supported, e.g. eight out of ten SIS100 buckets are filled by four SIS18 batches, each of two bunches. In addition, it should coordinate with the machine protection system, which protects the SIS100 and subsequent accelerators or experiments from beam induced damage. The beam transfer must be indicated for the beam instrumentation.